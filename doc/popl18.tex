\documentclass[format=acmsmall, review=true, anonymous, screen=true]{acmart}

\usepackage{agda}
\usepackage{catchfilebetweentags}
\usepackage{booktabs} % For formal tables
\usepackage{todonotes}

%%%%%%%%%% AGDA ALIASES

\newcommand{\APT}{\AgdaPrimitiveType}
\newcommand{\AK}{\AgdaKeyword}
\newcommand{\AM}{\AgdaModule}
\newcommand{\AS}{\AgdaSymbol}
\newcommand{\AStr}{\AgdaString}
\newcommand{\AN}{\AgdaNumber}
\newcommand{\AD}{\AgdaDatatype}
\newcommand{\AF}{\AgdaFunction}
\newcommand{\AR}{\AgdaRecord}
\newcommand{\ARF}{\AgdaField}
\newcommand{\AB}{\AgdaBound}
\newcommand{\AIC}{\AgdaInductiveConstructor}

\newcommand{\Set}{\mathbf{Set}}

\setlength\mathindent{0em}

% \usepackage[ruled]{algorithm2e} % For algorithms
% \renewcommand{\algorithmcfname}{ALGORITHM}
% \SetAlFnt{\small}
% \SetAlCapFnt{\small}
% \SetAlCapNameFnt{\small}
% \SetAlCapHSkip{0pt}
% \IncMargin{-\parindent}


% Metadata Information
\acmJournal{TWEB}
\acmVolume{9}
\acmNumber{4}
\acmArticle{39}
\acmYear{2010}
\acmMonth{3}
\copyrightyear{2009}
%\acmArticleSeq{9}

% Copyright
%\setcopyright{acmcopyright}
\setcopyright{acmlicensed}
%\setcopyright{rightsretained}
%\setcopyright{usgov}
%\setcopyright{usgovmixed}
%\setcopyright{cagov}
%\setcopyright{cagovmixed}

% DOI
\acmDOI{0000001.0000001}

% Paper history
\received{July 2017}
% \received[revised]{March 2009}
% \received[accepted]{June 2009}


\bibliographystyle{ACM-Reference-Format}
\citestyle{acmauthoryear}

% Document starts
\begin{document}
% Title portion. Note the short title for running heads 
\title[A Type and Scope Safe Universe of Syntaxes with Binding]{A Scope Safe Universe of Syntaxes with Binding, Their Semantics and Proofs}
\author{Guillaume Allais}
\affiliation{%
  \institution{Radboud University}
  \city{Nijmegen}
  \country{The Netherlands}}
\author{Robert Atkey}
\author{James Chapman}
\author{Conor McBride}
\affiliation{%
  \institution{University of Strathclyde}
  \city{Glasgow}
  \country{United Kingdom}
}
\author{James McKinna}
\affiliation{%
  \institution{University of Edinburgh}
  \department{LFCS}
  \city{Edinburgh}
  \country{United Kingdom}
}

\begin{abstract}
Syntaxes with binding are omnipresent in Programming Languages
research but also in the more practical setting of Embedded
Domain Specific Languages. The advanced features available in
some languages' type systems has made it possible to statically
enforce well-scopedness. However the user still has to write a
lot of boilerplate code to get common scope safe programs (e.g.
renaming, substitution, CPS transformation, printing with names,
etc.) and the proof that they are well-behaved.

Building on an abstract but nonetheless expressive notion of
semantics and a universe of syntaxes with binding, we demonstrate
how to (1) implement these traversals once and for all by generic
programming, and (2) how to derive their properties by generic proving.
So, the programmer will never again have to...

Building on an abstract but nonetheless expressive notion of
semantics and a universe of syntaxes with binding, we demonstrate
how to implement these traversals once and for all by generic
programming, and how to derive their properties by generic proving.
All of this work has been fully formalised in Agda and is available
at \fbox{REDACTED (see accompanying material)}.
\end{abstract}


%
% The code below should be generated by the tool at
% http://dl.acm.org/ccs.cfm
% Please copy and paste the code instead of the example below. 
%
\begin{CCSXML}
<ccs2012>
 <concept>
  <concept_id>10010520.10010553.10010562</concept_id>
  <concept_desc>Computer systems organization~Embedded systems</concept_desc>
  <concept_significance>500</concept_significance>
 </concept>
 <concept>
  <concept_id>10010520.10010575.10010755</concept_id>
  <concept_desc>Computer systems organization~Redundancy</concept_desc>
  <concept_significance>300</concept_significance>
 </concept>
 <concept>
  <concept_id>10010520.10010553.10010554</concept_id>
  <concept_desc>Computer systems organization~Robotics</concept_desc>
  <concept_significance>100</concept_significance>
 </concept>
 <concept>
  <concept_id>10003033.10003083.10003095</concept_id>
  <concept_desc>Networks~Network reliability</concept_desc>
  <concept_significance>100</concept_significance>
 </concept>
</ccs2012>  
\end{CCSXML}

\ccsdesc[500]{Computer systems organization~Embedded systems}
\ccsdesc[300]{Computer systems organization~Redundancy}
\ccsdesc{Computer systems organization~Robotics}
\ccsdesc[100]{Networks~Network reliability}

%
% End generated code
%

% We no longer use \terms command
%\terms{Design, Algorithms, Performance}

\keywords{Wireless sensor networks, media access control,
multi-channel, radio interference, time synchronization}


% \thanks{}


\maketitle

% The default list of authors is too long for headers}
\renewcommand{\shortauthors}{G. Allais, R. Atkey, J. Chapman, C. McBride, J. McKinna}

\begin{abstract}
Syntaxes with binding are omnipresent in Programming Languages
research but also in the more practical setting of Embedded
Domain Specific Languages. The advanced features available in
some languages' type systems has made it possible to statically
enforce well-scopedness. However the user still has to write a
lot of boilerplate code to get common scope safe programs (e.g.
renaming, substitution, CPS transformation, printing with names,
etc.) and the proof that they are well-behaved.

Building on an abstract but nonetheless expressive notion of
semantics and a universe of syntaxes with binding, we demonstrate
how to implement these traversals once and for all by generic
programming, and how to derive their properties by generic proving.
\end{abstract}

\section{Introduction}

Nowadays the software programmer developping an embedded DSL~\cite{hudak1996building}
and the PL researcher formalising a calculus both know and
leverage the host language's type system. Using Generalised
Algebraic Data Types (GADTs) or the more general indexed
families of Type Theory~\cite{dybjer1994inductive} for their deep embedding, they can
\emph{statically} enforce some of the invariants present in
their language. Managing the scope is a popular use case~\cite{altenkirch1999monadic} as
directly manipulating raw de Bruijn indices is error-prone.

This paper starts with primers on scope safe terms, scope preserving
programs acting on them and a generic way to represent data types.
These introductory sections help us build an understanding of the
problem at hand as well as a toolkit that leads us to the original
content of this paper: a universe of scope safe syntaxes with binding
together with a generic notion of scope safe semantics for these syntaxes.
This give us the opportunity to write generic implementations of renaming,
substitution but also elaboration of a surface language to a core one,
and normalisation by evaluation. We also explore other opportunities for
generic programming: the interpretation of binding as self-reference leads
to the ability to give a finite representation of potentially infinite
objects.
\todo{proofs}

\section{A Primer on Scope Safe Terms}\label{section:primer-term}

Scope safe terms are following a strict discipline which enforces statically
that they may only refer to variables introduced by a binder beforehand. A
scope safe language is a programming language in which all the valid terms
are guaranteed to be scope safe.

Bellegarde and Hook~\citeyear{BELLEGARDE1994287}, Bird and Patterson~\citeyear{bird_paterson_1999},
and Altenkirch and Reus~\citeyear{altenkirch1999monadic} introduced the
nowadays classic presentation of scope safe languages using inductive
\emph{families}~\cite{dybjer1994inductive} to track scoping information
at the type level. Instead of describing the type of abstract syntax trees
of the language as the fixpoint of an endofunctor on $\Set{}$, they used
an endofunctor on $\Set{}^{\Set{}}$ where the $\Set{}$ index corresponds
to the set of variables in scope. Because the empty Set has no inhabitant,
it is a natural representation of the empty scope. Conversely, the functor
$M(X) = 1 + X$ is used to extend the running scope with an extra variable.

This generic presentation of scope safe languages leads to the following
definition of the untyped $\lambda$-calculus. The endofunctor
$T(F) = \lambda X \in \Set{}. X + (F(X) \times F(X)) + F(1 + X)$
offers a choice of three constructors. The first one corresponds to the variable
case; it packages an inhabitant of $X$, the index $\Set{}$. The second corresponds
to an application node; both the function and its argument live in the same
scope as the overall expression. Last but not least, the third corresponds to
a $\lambda$-abstraction; it extends the current scope with a fresh variable.
The language is obtained as the fixpoint of $T$:
\[
   \mathit{Lam} = \mu F \in \Set{}^{\Set{}}.
   \lambda X \in \Set{}. X + (F(X) \times F(X)) + F(1 + X)
\]
The proof that the fixpoint is functorial then corresponds to renaming
whilst the proof that it is monadic implements substitution: the variable
constructor is return and bind defines parallel substitution.

\subsection{A Mechanized Variant of Altenkirch and Reus' Untyped Calculus}

There is no reason to restrict this technique to fixpoints of endofunctors
on $\Set{}^{\Set{}}$ apart from the fact that renaming and substitution
correspond to well-known structures in that specific case. The more general
case of fixpoints of (strictly positive) endofunctors on $\Set{}^I$ can be
endowed with similar operations by using what Altenkirch, Chapman and
Uustalu~\citeyear{Altenkirch2010, JFR4389} refer to as relative monads.

In this paper, we pick $I = \mathbb{N}$ where the natural number used as
an index is straightforwardly the number of (de Bruijn) variables in scope.
This natural number can be seen as a list associating to each variable in
scope an element of the unit type; in a typed setting, it would carry the
variable's type instead.

Our implementation language is Agda~\cite{norell2009dependently} however
these techniques are language independent: any dependently typed language
whose logic is at least as powerful as Martin-L\"of Type
Theory~\cite{martin1982constructive} equipped with inductive
families~\cite{dybjer1994inductive} ought to do.

In order to lighten the presentation, we weaponise the observation that the
current scope is either threaded to subterms (e.g. in the application's case)
or adjusted (e.g. in the $\lambda$-abstraction's case) by introducing combinators
to build indexed types. Although it may seem surprising at first to define
infix operators of arity three, they are meant to be used partially applied,
surrounded by \AF{[\_]} which turns an indexed Set into a Set by implicitly
quantifying over the index. The first two combinators are the pointwise liftings
of the function space and the product type respectively, both silently threading
the underlying scope. The third one is simply the constant function turning a Set
into an indexed Set by ignoring the index. Finally, the last one makes explicit
the \emph{adjustment} made to the index by a function. The use of $\_\vdash{}\_$\footnote{
In Agda, underscores in a function's name denote the positions in which the
function's arguments are to be inserted.} for the latter is meant to suggest
its connection to the mathematical convention of only mentioning context
\emph{extensions} when presenting judgements (see e.g. \cite{martin1982constructive}).

\begin{figure}[h]
\begin{minipage}{0.45\textwidth}
\ExecuteMetaData[indexed.tex]{arrow}
\end{minipage}\hspace{2em}
\begin{minipage}{0.45\textwidth}
\ExecuteMetaData[indexed.tex]{product}
\end{minipage}

\begin{minipage}{0.25\textwidth}
\ExecuteMetaData[indexed.tex]{constant}
\end{minipage}
\begin{minipage}{0.40\textwidth}
\ExecuteMetaData[indexed.tex]{adjust}
\end{minipage}\hspace{2em}
\begin{minipage}{0.25\textwidth}
\ExecuteMetaData[indexed.tex]{forall}
\end{minipage}
\caption{Combinators to build indexed Sets}\label{figure:indexed}
\end{figure}

For instance, the fairly compact expression
\AF{[} \AF{suc} \AF{⊢} (\AB{P} \AF{∙×} \AB{Q}) \AF{⟶} \AB{R} \AF{]}
corresponds to the more verbose type
\AS{∀} \{\AB{i}\} \AS{→} (\AB{P} (\AF{suc} \AB{i}) \AF{×} \AB{Q} (\AF{suc} \AB{i})) \AS{→} \AB{R} \AB{i}.
Using these combinators, the untyped $\lambda$-calculus can be represented
using the following definitions:

\begin{figure}[h]
\begin{minipage}{0.45\textwidth}
\ExecuteMetaData[var.tex]{var}
\end{minipage}\hspace{2em}
\begin{minipage}{0.45\textwidth}
\ExecuteMetaData[motivation.tex]{tm}
\end{minipage}
\caption{Scope Aware Variables and Untyped $\lambda$-Terms\label{scoped-untyped}}
\end{figure}

The inductive family \AD{Var} corresponds to well scoped de Bruijn~\citeyear{de1972lambda}
indices. Its first constructor (\AIC{z} for zero) states that we have a name to refer to
the nearest binder in a non-empty scope. The second one (\AIC{s} for successor) lifts a
name for a variable in a given scope into a name for it in the extended scope where
an extra variable has been bound. Both of their types have been written using combinators,
altought we will abstain from unfolding them in the future, we do so here in the hope
it will help the reader get acquainted with them: they respectively normalise to
$\forall n. \mathbf{Var}(\mathit{suc}(n))$ for \AIC{z},
and $\forall n. \mathbf{Var}(n) \rightarrow \mathbf{Var}(\mathit{suc}(n))$ for \AIC{s}.

The \AD{ℕ}-indexed family \AD{Lam} is the variant of Altenkirch and Reus' untyped
$\lambda$-calculus. The two interesting constructors are the one lifting variables
to terms and the $\lambda$-abstraction whose body lives in an extended context.

\section{A Primer on Scope Safe Programs}\label{section:primer-program}

This scope safe deep embedding of the untyped $\lambda$-calculus is
naturally only a start: once the programmer has access to a good
representation of the language she is interested in, she wants and
needs to (re)implement standard traversals manipulating terms.
Renaming and substitution are perhaps the two most iconic examples
of such traversals. And now that well-scopedness is enforced in
the terms' indices, all of these traversals have to be implemented
in a scope safe manner. These constraints show up in the type of
renaming and substitution which can be defined as follows:

\begin{figure}[h]
\begin{minipage}{0.45\textwidth}
\ExecuteMetaData[motivation.tex]{ren}
\end{minipage}\hspace{2em}
\begin{minipage}{0.45\textwidth}
\ExecuteMetaData[motivation.tex]{sub}
\end{minipage}
\caption{Scope Preserving Renaming and Substitution}
\end{figure}

Looking more closely at these two functions' code, it is quite evident
that they have a very similar structure. In each case, we have used
two (function specific) auxiliary definitions named \AF{⟦V⟧} and \AF{extend}
respectively to highlight this fact. Abstracting away this shared structure
would allow for these definitions to be refactored, and their common
properties to be proved in one swift move.

Previous efforts in dependently typed
programming~\cite{mcbride2005type,benton2012strongly,allais2017type}
have precisely achieved this goal and refactored renaming and substitution,
but also normalisation by evaluation, printing with names or CPS conversion
as various instances of a more general traversal. Unpublished results also
demonstrate that typechecking in the style of Atkey~\citeyear{atkey2015algebraic}
fits in that framework. To be able to make sense of this body of work, we
need to introduce three new definitions:

A \AF{Thinning} from $m$ to $n$ is a function from \AD{Var} $m$ to
\AD{Var} $n$. It is a special case of a notion of environment that
stores values living in a scope $n$ for each variable in a scope $m$.
We introduce environment as records (this helps the host language's
type inference reconstruct the type of values) and write
(\AB{m} \AR{─Env}) \AB{𝓥} \AB{n} for such an environment with values
in \AB{𝓥}.

\begin{figure}[h]
\begin{minipage}{0.55\textwidth}
\ExecuteMetaData[environment.tex]{env}
\end{minipage}\hspace{2em}
\begin{minipage}{0.35\textwidth}
\ExecuteMetaData[environment.tex]{thinning}
\end{minipage}
\caption{Environments of Well Scoped Values and Thinnings}
\end{figure}

\AF{Thinning}s subsume more structured notions such as the Category of
Weakenings~\cite{altenkirch1995categorical} or Order Preserving
Embeddings~\cite{chapman2009type}. In particular, it does not prevent the
user from defining arbitrary permutations or from introducing contractions
although we will not use such instances. The fact that our representation
is a function space grants us monoid laws ``for free'' as per Jeffrey's
observation~\citeyear{jeffrey2011assoc}.

The \AF{□} combinator turns any \AD{ℕ}-indexed Set into one that can absorb
thinnings. It is akin to Kripke-style quantification over all possible future
worlds and \AF{□} (\AB{D} \AF{→} \AB{D}) indeed corresponds to the Kripke
function space used in normalisation by evaluation via a domain \AB{D}.
Because we can define an identity \AF{Thinning} and \AF{Thinning}s do
compose, \AF{□} is a comonad.

The notion of \AF{Thinnable} is the property of being stable under thinnings,
in other words: \AF{□} is a comonad and \AF{Thinnable}s are its coalgebras.
It is a crucial property for values to have if one wants to be able to push
them under binders. Unsurprisingly, from the comonadic structure we get that
the \AF{□} combinator freely turns any \AD{ℕ}-indexed Set into a \AF{Thinnable}
one.

\begin{figure}[h]
\begin{minipage}{0.45\textwidth}
\ExecuteMetaData[environment.tex]{box}
\ExecuteMetaData[environment.tex]{comonad}
\end{minipage}\hspace{2em}
\begin{minipage}{0.45\textwidth}
\ExecuteMetaData[environment.tex]{thinnable}
\ExecuteMetaData[environment.tex]{freeth}
\end{minipage}
\caption{The \AF{□} comonad, Thinnable, and the free Thinnable.}
\end{figure}

Equipped with these new notions, we can define an abstract
concept of semantics for our scope safe language. Broadly
speaking, a semantics turns our deeply embedded abstract
syntax trees into the shallow embedding of the corresponding
parametrised higher order abstract syntax term. We get various
semantics by using different 'host languages' for this shallow
embedding. A semantics with values in $\mathcal{V}$ and
computations in $\mathcal{C}$ is meant to give rise to a
traversal which, provided a term and an environment of values
for each one of the variables in scope, delivers a computation.
The traversal \AF{sem} realises this specification generically
for all Semantics.

A Semantics is characterised by a set of constraints. First of all,
values should be thinnable so that \AF{sem} may push the environment
under binders. Second, the set of computations needs to be closed
under various combinators which are the semantical counterparts of
the language's constructors. Here the semantical counterpart of
application is not particularly interesting. However the interpretation
of the $\lambda$-abstraction is of interest: it is a variant on
the Kripke function space one can find in normalisation by evaluation.
In all possible thinnings of the scope at hand, it promise to deliver
a computation whenever it is provided with a value for its newly
bound variable. This is concisely expressed by the type
(\AF{□} (\AB{\mathcal{V}} \AF{→} \AB{\mathcal{C}})).

\begin{figure}[h]
\begin{minipage}{0.45\textwidth}
\ExecuteMetaData[motivation.tex]{rsem}
\end{minipage}\hspace{2em}
\begin{minipage}{0.45\textwidth}
\ExecuteMetaData[motivation.tex]{sem}
\end{minipage}
\caption{Semantics for Lam and their Fundamental Lemma}
\end{figure}

Coming back to renaming and substitution, we can see that they do fit
in the \AR{Sem} framework. We also include the definition of a very
basic printer relying on a name supply to highlight the fact that
computations can very well be effectful. Both Printing and Renaming
highlight the importance of having a distinct notion of values and
computations: the type of values in their respective environments
are distinct from their type of computations.

\begin{figure}[h]
\begin{minipage}{0.40\textwidth}
\ExecuteMetaData[motivation.tex]{semren}
\ExecuteMetaData[motivation.tex]{semsub}
\end{minipage}\hspace{1em}
\begin{minipage}{0.50\textwidth}
\ExecuteMetaData[motivation.tex]{semprint}
\end{minipage}
\caption{Renaming, Substitution and Printing as Instances of \AR{Sem}}
\end{figure}

All of these examples are already desribed at length by Allais, Chapman,
McBride and McKinna~\citeyear{allais2017type} so we will not spend any
more time on them. They have also obtained the simulation and fusion
theorems demonstrating that these traversals are well-behaved as
corollaries of more general results expressed in terms of that generic
traversal. We will come back to this in Section~\ref{section:simulation}.

One important observation to make is the tight connection between the
constraint described by \AR{Sem} and the definition of \AD{Lam}: the
semantical combinators are related to the corresponding constructors
where the recursive occurences of the inductive family have been replaced
with either a computation or a Kripke function space whenever an
extra variable was bound. This suggest that it ought to be possible
to compute the definition of \AF{Sem} from the one of the datatype.

\section{A Primer on the Universe of Data Types}

Chapman, Dagand, McBride and Morris~\citeyear{Chapman:2010:GAL:1863543.1863547}
defined a universe of data types inspired by Dybjer and Setzer's
finite axiomatisation of Inductive-Recursive definitions~\citeyear{Dybjer1999}.
This explicit definition of \emph{codes} for data types empowers the
user to write generic programs tackling \emph{all} of the data types
one can obtain this way. In this section we recall the main aspects
of this construction we are interested in to build up our generic
representation of syntaxes with binding.

The first component of this universe's definition is an inductive type
of \AD{Desc}riptions of strictly positive functors on $\Set{}$. It has
three constructors: one to store data (the rest of the description can
depend upon this stored value), one to attach a recursive substructure
and one to stop.

These constructors give the programmer the ability to build up the data
types she is used to. For instance, the functor corresponding to lists
of elements in $A$ stores one bit of data: whether the current node is
the empty list or not. Depending on that bit, the rest of the description
is either the ``stop'' token or a pair of an element in $A$ and a
recursive substructure (i.e. the tail of the list).

\begin{figure}[h]
\begin{minipage}{0.45\textwidth}
\ExecuteMetaData[generic-data.tex]{desc}
\end{minipage}\hspace{2em}
\begin{minipage}{0.45\textwidth}
\ExecuteMetaData[generic-data.tex]{listD}
\end{minipage}
\caption{Data Descriptions and List Description}\label{figure:listD}
\end{figure}

The recursive function \AF{⟦\_⟧} makes the interpretation of the
Descriptions formal by associating a value in $\Set{}^\Set{}$ to
each one of them. They essentially give rise to unit terminated
right nested tuples. Together with \AF{⟦\_⟧}, we can define the
\AF{fmap} recursive function witnessing the fact that the meaning
of a description is indeed functorial. This is the first example
of generic programming over all the functors one can obtain as
the meaning of a description.

\begin{figure}[h]
\begin{minipage}{0.35\textwidth}
\ExecuteMetaData[generic-data.tex]{interp}
\end{minipage}\hspace{2em}
\begin{minipage}{0.55\textwidth}
\ExecuteMetaData[generic-data.tex]{fmap}
\end{minipage}
\caption{Meaning of Descriptions and Proof of Functoriality}
\end{figure}

Because all of the functors obtained as meanings of descriptions are
strictly positive, it is possible to take their least fixpoint \AD{μ}.
We index this inductive definition of the least fixpoint with a
size~\cite{DBLP:journals/corr/abs-1012-4896} so that functions such
as \AF{fold}, the proof that \AD{μ} \AB{d} is the initial algebra
corresponding to \AB{d}'s meaning, are seen as obviously terminating
by the system thanks to a type-based argument. Indeed, the use of
\AF{fmap} obscures the fact that recursive calls are only performed
on strict subterms and without a type-based approach this concise
definition would be rejected.

\begin{figure}[h]
\begin{minipage}{0.45\textwidth}
\ExecuteMetaData[generic-data.tex]{mu}
\end{minipage}\hspace{2em}
\begin{minipage}{0.45\textwidth}
\ExecuteMetaData[generic-data.tex]{fold}
\end{minipage}
\caption{Fixpoint and Generic Fold}
\end{figure}

This demonstrates that this approach allows us to define generically
the iteration principle associated to all the datatypes which arise
as the fixpoint of a description's meaning. It seems appropriate to
base our universe of scope safe syntaxes on a similar construction
so that we may be able to define generically a notion of semantics
for all the syntaxes with binding one may come up with. Chapman,
Dagand, McBride and Morris also give a more general universe which
supports higher order branching (but still does not have a notion
of variable). We decidedly stick to finitary constructors, thus
sticking to the common understanding of `syntax'.

\section{A Universe of Scope Safe Syntaxes}

Our universe of scope safe syntaxes follows the same principle
as McBride's universe of datatypes except that we are not building
endofunctors on $\Set{}$ but rather $\Set{}^ℕ$. Descriptions can
be built using three constructors: the first one makes it possible
to store data (and, as usual, the rest of the description may
depend upon the value stored), the second takes a natural number
$m$ and corresponds to a substructure with exactly $m$ additional
variable in scope and the last one ends the definition.

The meaning function \AF{⟦\_⟧} we associate to a description is not
quite an endofunctor on $\Set{}^ℕ$; it is more general than that.
Given an $X$ that interprets substructures with an extra $m$ bound
variables in a scope that has $n$ bound variables already as $X\,m\,n$,
we give a description a meaning as an $\Set{}^ℕ$. The astute reader
may have noticed that \AF{⟦\_⟧} is uniform in $X$ and $n$; however
refactoring \AF{⟦\_⟧} following this observation would lead to a
definition harder to use with the combinators for indexed sets
described in Figure~\ref{figure:indexed} which make our types
much more readable.

\begin{figure}[h]
\begin{minipage}{0.45\textwidth}
\ExecuteMetaData[generic-syntax.tex]{desc}
\end{minipage}\hspace{2em}
\begin{minipage}{0.45\textwidth}
\ExecuteMetaData[generic-syntax.tex]{interp}
\end{minipage}
\caption{Syntax Descriptions and their Meaning as Functor}
\end{figure}

If we pre-compose the meaning function \AF{⟦\_⟧} with a notion
of `scope' (denoted \AF{Scope} here) which turns any \AD{ℕ}-indexed
family into a function of type \AD{ℕ} \AS{→} \AD{ℕ} \AS{→} $\Set{}$
by simply summing the two indices, we recover an endofunctor on
$\Set{}^ℕ$ and we can take its fixpoint. This time, instead
of considering the initial algebra, we opt for the free relative
monad~\cite{JFR4389}: the \AIC{`var} constructor corresponds to
return and we will define bind in the next section.

\begin{figure}[h]
\begin{minipage}{0.55\textwidth}
\ExecuteMetaData[generic-syntax.tex]{mu}
\end{minipage}\hspace{2em}
\begin{minipage}{0.35\textwidth}
\ExecuteMetaData[generic-syntax.tex]{scope}
\end{minipage}
\caption{Term Trees: The Free Relative Monads on Descriptions}
\end{figure}

Coming back to the well-scoped untyped $\lambda$-calculus,
we now have the ability to give its code. The variable
constructor will be introduced by the free monad construction
so we only have to describe two cases:
the application case where we have two substructures which do
not bind any extra argument and the $\lambda$-abstraction case
which has exactly one substructure with precisely one extra
bound variable.

We can then define constructors corresponding
to the original ones in Figure~\ref{scoped-untyped}: \AF{`V}
for \AIC{V} the variable constructor, \AF{`A} for \AIC{A} the
application one and \AF{`L} for \AIC{L} the $\lambda$-abstraction.
In Agda, we can actually almost define these as
pattern synonyms (the language currently does not allow the user
to specify type annotations for her pattern synonyms), meaning
that the end user can seemlessly write pattern-matching programs
on encoded terms without dealing with the gnarly details
of the encoding.

\begin{figure}[h]
\begin{minipage}{0.55\textwidth}
  \ExecuteMetaData[generic-syntax.tex]{LCD}
\end{minipage}
\begin{minipage}{0.35\textwidth}
  \ExecuteMetaData[generic-syntax.tex]{LC}
\end{minipage}

\begin{minipage}{0.30\textwidth}
  \ExecuteMetaData[generic-syntax.tex]{var}
\end{minipage}
\begin{minipage}{0.30\textwidth}
  \ExecuteMetaData[generic-syntax.tex]{app}
\end{minipage}\hspace{2em}
\begin{minipage}{0.30\textwidth}
  \ExecuteMetaData[generic-syntax.tex]{lam}
\end{minipage}
\caption{Example: The Untyped Lambda Calculus}
\end{figure}

It is the second time (the first time being the definition of
\AF{listD} in Figure~\ref{figure:listD}) that we use a
\AF{Bool} to distinguish between two constructors. In order
to avoid re-encoding the same logic,
the next section introduces combinators demonstrating that
descriptions are closed under sums and products.

\subsection{Common Combinators and Their Properties}\label{desccomb}

As we have seen previously, we can take the coproduct of two
descriptions by using a dependent pair whose first component
stores a \AF{Bool}ean tagging which branch was taken whilst
the second one uses that information to return the description
corresponding to that branch. We can define an appropriate
eliminator \AF{case} which given two continuations picks the
one corresponding to the chosen branch.

\begin{figure}[h]
\begin{minipage}{0.45\textwidth}
  \ExecuteMetaData[generic-syntax.tex]{sumcomb}
\end{minipage}\hspace{2em}
\begin{minipage}{0.45\textwidth}
  \ExecuteMetaData[generic-syntax.tex]{case}
\end{minipage}
\caption{Descriptions are closed under Sums}
\end{figure}

Closure under product is however a bit more technical: it is
defined by induction on the first description of the product.
The definition is purely structural except for the ``stop''
constructor which gets replaced by the second description. Because
of the indirect nature of the definition of closure under
products, it is convenient to have functions going back and
forth between the interpretation of a product of descriptions
and the product of their respective interpretations.

\begin{figure}[h]
\begin{minipage}{0.45\textwidth}
  \ExecuteMetaData[generic-syntax.tex]{paircomb}
\end{minipage}\hspace{2em}
\begin{minipage}{0.45\textwidth}
  \ExecuteMetaData[generic-syntax.tex]{pairunpair}
\end{minipage}
\caption{Descriptions are closed under Products}
\end{figure}


\section{Generic Scope Safe Programs for Syntaxes}\label{section:semantics}

Based on the structure made explicit in the example worked out
in Section~\ref{section:primer-program}, we can define a generic notion of
semantics for all syntax descriptions. It is once more parametrised
by two \AD{ℕ}-indexed families \AB{𝓥} and \AB{𝓒} corresponding
respectively to values associated to bound variables and
computations delivered by evaluating terms. These two families
have to abide by three constraints
\begin{itemize}
\item Values should be thinnable for us to be able to push the
      evaluation environment under binders;
\item Values should embed into computations for us to be able
      to the return the value associated to a variable as the
      result of its evaluation;
\item Last but not least, we should have an algebra turning
      a term where substructures have been replaced with
      either computations or kripke functional spaces (depending
      on whether extra bound variables have been introduced)
      into computations
\end{itemize}
Here we crucially use the fact that the meaning of a description is
defined in terms of a function interpreting substructures which has
the type \AD{ℕ} \AS{→} \AD{ℕ} \AS{→} $\Set{}$, i.e. that gets access
to the current scope but also the exact number of newly-bound variables.
We define such a function, \AF{Kripke}, by case analysis on the number
of newly-bound variables: if it's $0$ then we expect the substructure
to simply be a computation (the result of the evaluation function's
recursive call) but if there are newly bound variables then we expect
a function which takes one value for each one of them and delivers
a computation corresponding to the evaluation of the body of the binder
in the extended environment.

\begin{figure}[h]
\begin{minipage}{0.45\textwidth}
  \ExecuteMetaData[environment.tex]{kripke}
\end{minipage}\hspace{2em}
\begin{minipage}{0.45\textwidth}
  \ExecuteMetaData[generic-syntax.tex]{semantics}
\end{minipage}
\caption{A Generic Notion of Semantics}
\end{figure}

It is once more the case that the abstract notion of Semantics comes
with a fundamental lemma: all \AD{ℕ}-indexed families \AB{𝓥} and
\AB{𝓒} satisfying the three criteria we have put forward give rise
to an evaluation function. Here the fundamental lemma is called \AF{sem}
and it is defined mutually with a function \AF{body} turning a \AF{Scope}
(i.e. a substructure in a potentially extended context) into a \AF{Kripke}
(i.e. a subcomputation expecting a value for each newly bound variable).

\begin{figure}[h]
{\center \ExecuteMetaData[generic-syntax.tex]{sembody}}
\hspace{-1em}\begin{minipage}{0.40\textwidth}
\ExecuteMetaData[generic-syntax.tex]{sem}
\end{minipage}\hspace{2em}
\hspace{-1em}\begin{minipage}{0.50\textwidth}
\ExecuteMetaData[generic-syntax.tex]{body}
\end{minipage}
\caption{Fundamental Lemma of \AR{Sem}antics}
\end{figure}

Renaming can be defined generically for all syntax descriptions as a
semantics with \AF{Var} as values and \AD{Tm} as computations. The
two first constraints on \AF{Var} described earlier are trivially
satisfied. Because renaming strictly respects the structure of the
term it goes through, the algebra is implemented using \AF{fmap}.
When dealing with the body a binder, we simply `reify' the
\AF{Kripke} function by evaluating it in an extended context and
feeding it dummy values corresponding to the extra variables
introduced by that context. This is reminiscent both of what we
did in Section~\ref{section:primer-program} and the definition
of reification in the setting of normalisation by evaluation
(see e.g. Coquand's work~\citeyear{coquand2002formalised}).

Substitution can be defined in a similar manner. Of the two
constraints applying to terms as values, the first one corresponds
precisely to renaming and the second one is trivial. The algebra
can once more be defined by using \AF{fmap} and reifying the bodies
of binders. This reification process can be implemented generically
for semantics which have ``VarLike'' values i.e. values that are
thinnable and such that we can craft dummy ones in non-empty contexts.

\begin{figure}[h]
\begin{minipage}{0.40\textwidth}
  \ExecuteMetaData[generic-syntax.tex]{varlike}
\end{minipage}\hspace{2em}
\begin{minipage}{0.50\textwidth}
  \ExecuteMetaData[generic-syntax.tex]{reify}
\end{minipage}

\begin{minipage}{0.45\textwidth}
  \ExecuteMetaData[generic-syntax.tex]{renaming}
\end{minipage}\hspace{2em}
\begin{minipage}{0.45\textwidth}
  \ExecuteMetaData[generic-syntax.tex]{substitution}
\end{minipage}
\caption{Generic Renaming and Substitution for All Scope Safe Syntaxes with Binding}
\end{figure}


\section{Other Generic Programs}

\subsection{Sugar and Desugaring as a Semantics}

One of the advantages of having a universe of programming languages
descriptions is the ability to concisely define an \emph{extension}
of an existing language by using \AD{Desc}ription transformers
grafting extra constructors à la Swiestra~\citeyear{swierstra_2008}.
This is made extremely simple by the
disjoint sum combinator \AF{\_`+\_} which we have already seen
in Section~\ref{desccomb}.
An example of such an extension is the addition of let-bindings to
an existing language. Let bindings allow the user to avoid repeating
herself by naming sub-expressions and then using these names to refer
to the associated terms.

\begin{figure}[h]
\begin{minipage}{0.45\textwidth}
  \ExecuteMetaData[generic-syntax.tex]{ntimes}
\end{minipage}\hspace{2em}
\begin{minipage}{0.45\textwidth}
  \ExecuteMetaData[generic-syntax.tex]{letcode}
\end{minipage}
\caption{Parallel Let Binding}
\end{figure}

In a dependently typed language a type may depend on a value which
in the presence of let bindings may be a variable standing for an
expression. The user naturally does not want it to make any difference
whether she used a variable referring to a let-bound expression or
the expression itself. Various typechecking strategies can accomodate
this expectation: in Coq~\cite{Coq:manual} let bindings are primitive
constructs of the language and have their own typing and reduction
rules whereas in Agda they are elaborated away to the core language
by inlining.

This latter approach to extending a language \AB{d} with let bindings
by inlining them before typechecking can be implemented generically as
a Semantics over (\AF{Let} \AF{`+} \AB{d}) where values in the environment and
computations both are let-free terms. The algebra of that semantics can
be defined by parts: the old constructors are simply interpreted using
the algebra defined generically for the \AF{Substitution} semantics whilst
the newer one precisely provides the extra values to be added to the
environment (we leave the definition of \AF{alg'} out because of a lack
of space). The process of removing let binders is kickstarted with a
dummy environment associating each variable to itself.

\begin{figure}[h]
\begin{minipage}{0.50\textwidth}
  \ExecuteMetaData[generic-syntax.tex]{unletcode}
\end{minipage}\hspace{2em}
\begin{minipage}{0.40\textwidth}
  \ExecuteMetaData[generic-syntax.tex]{unlet}
\end{minipage}
\caption{Inlining Let Binding}
\end{figure}

In about 20 lines of code we have defined a generic extension of
syntaxes with binding together with a semantics which corresponds
to an elaborator translating away this new constructor. In their
own setting working on a given language, Allais, Chapman, McBride
and McKinna~\citeyear{allais2017type} have shown that it is similarly
possible to implement a Continuation Passing Style transformation as
a semantics.

The ease with which one can define such generic transformations
suggests that this setup could be a good candidate to implement
generic compilation passes.

\subsection{(Unsafe) Normalisation by Evaluation}

A key type of traversal we have not studied yet is a language's
evaluator. Our universe of syntaxes with binding does not impose
any typing discipline on the user defined languages and as such
cannot guarantee their totality. This is embodied by our running
example: the untyped $\lambda$-calculus. As a consequence there
is no hope for a safe generic framework to define normalisation
functions.

The clear connection between the \AF{Kripke} functional space
characteristic of our semantics and the one that shows up in
normalisation by evaluation suggests we ought to manage to
give a generic framework for normalisation by evaluation.
By temporarily \textbf{disabling Agda's positivity checker},
we can define a generic reflexive domain \AD{Dm} in which to
interpret our syntaxes. It has three constructors corresponding
respectively to a free variable, a constructor's counterpart where
scopes have become \AF{Kripke} functional spaces on \AD{Dm} and
an error token because the evaluation of untyped programs may
(and usually does!) go wrong.

\begin{figure}[h]
{\center \ExecuteMetaData[generic-syntax.tex]{domain}}
\caption{Generic Reflexive Domain}
\end{figure}

This datatype definition is utterly unsafe. The more conservative
user will happily restrict herself to typed settings where the
domain can be defined as a logical predicate or opt instead for
a step-indexed approach. But this domain does make it possible
to define a generic \AF{nbe} semantics as well as a reification
function turning elements of the reflexive domain into terms.
By composing them, we obtain the normalisation function which
gives its name to normalisation by evaluation.

The user still has to explicitly pass an interpretation of
the various constructors because there is no way for us to
know what the binders are supposed to represent: they may
stand $\lambda$-abstractions, $\Sigma$-types, fixpoints, or
anything else she may want to define.


\begin{figure}[h]
\ExecuteMetaData[generic-syntax.tex]{nbe-setup}
\caption{Generic Normalisation by Evaluation Framework}
\end{figure}

Using this setup, we can write a normaliser for the untyped
$\lambda$-calculus: we use \AF{case} to distinguish between
the semantical counterpart of the application constructor on
one hand and the $\lambda$-abstraction one on the other.
The latter is trivial: functions are already
values! The semantical counterpart of application proceeds by
case analysis on the function: if it corresponds to a
$\lambda$-abstraction, we can fire the redex by using the kripke
functional space; otherwise we grow the spine of stuck
applications.


\begin{figure}[h]
\ExecuteMetaData[generic-syntax.tex]{nbelc}
\caption{Normalisation by Evaluation for the Untyped $\lambda$-Calculus}
\end{figure}


\section{Building Generic Proof about Generic Programs}\label{section:simulation}

Allais, Chapman, McBride, and McKinna~\citeyear{allais2017type} have
already shown that, for their specific language, introducing an abstract
notion of Semantics not only reveals the shared structure of common
traversals, it also allows them to give abstract proof frameworks for
simulation or fusion lemmas. Their idea naturally extends to our generic
presentation of semantics for all syntaxes.

The most important concept in this section is (\AF{Zip} \AB{d}), a relation
transformer which characterises structurally equal layers such that their
substructures are themselves related by the relation it is passed as an
argument. It is defined by induction on the description and case analysis
on the two layers which are meant to be equal. It inherits a lot of its
relational arguments' properties: whenever \AB{R} is reflexive (respectively
symmetric or transitive) so is \AF{Zip} \AB{d} \AB{R}.

\begin{figure}[h]
 \ExecuteMetaData[generic-simulation.tex]{ziptype}
\caption{Zip: Characterising Structurally Equal Values with Related Substructures}
\end{figure}

\subsection{Simulation Lemma}

A \AF{Zip} constraint appears naturally when we want to say that a
semantics can simulate another one. Given a relation
%\AB{𝓡^𝓥}
relating values in \AB{𝓥_1} and \AB{𝓥_2}, and a relation
% \AB{𝓡^𝓒}
relating computations in \AB{𝓒_1} and \AB{𝓒_2}, we can define
a relation \AF{Kripke^R} relating values \AF{Kripke} \AB{𝓥_1} \AB{𝓒_1}
and \AF{Kripke} \AB{𝓥_2} \AB{𝓒_2} by stating that they send
related inputs to related outputs. We can then use \AF{Zip} to
express the idea that two semantic objects of respective types
\AF{⟦} \AB{d} \AF{⟧} (\AF{Kripke} \AB{𝓥_1} \AB{𝓒_1}) and
\AF{⟦} \AB{d} \AF{⟧} (\AF{Kripke} \AB{𝓥_2} \AB{𝓒_2}) are
synchronised.

The formal definition of one semantics being able to simulate another
can then state (among other constraints) that the \ARF{alg}ebras have
to send synchronised semantic objects to related computations. We can
prove a generic theorem showing that for each pair of semantics respecting
the semantics, we get related computations given related inputs.
Instantiating this generic simulation lemma, we can for instance get
that Renaming is a special case of Substitution.

\begin{figure}[h]
 \ExecuteMetaData[generic-simulation.tex]{rensub}
\caption{Renaming can be Simulated using Substitution}
\end{figure}

\section{Conclusion}

We have started from an example of a scope safe language (the
untyped $\lambda$-calculus), have studied various common traversals
and noticed their similarity. After introducing a notion of semantics
and refactoring these traversals as various instances of the same
fundamental lemma, we have observed the tight connection between the
abstract definition of semantics and the shape of the language. By
extending a universe of datatype descriptions to support a notion of
binding, we have managed to give a generic presentation of syntaxes
with binding as well as a large class of scope safe programs acting
on them: from Renaming and Substitution, to Normalisation by Evaluation,
and Elaboration to a core language. We have also seen how this setup
could be applied to a different domain: the representation of potentially
cyclic structures by finite artefacts. Last but not least, we have
quickly seen how to construct proofs about these generic programs.
The diverse influences leading to this body of work suggest many
opportunities for future research:

The question of a universe of syntaxes with binding which are not only
well scoped but also intrinsically well typed by construction is an
exciting challenge. The existing variation on the universe of datatypes
giving a universe of inductive families~\cite{dybjer1994inductive}
is a natural candidate.

Our example of the elaboration of an enriched language to a core one,
and Allais, Chapman, McBride and McKinna's implementation of a
Continuation Passing Style conversion function begs the question of how
many such common compilation passes can be implemented generically.
An extension of McBride's theory of ornaments~\citeyear{mcbride2010ornamental}
could provide an appropriate framework to highlight the connection
between various languages, some being seen as the extension of others.

The potentially cyclic structures we have studied have been improved
upon by Hamana~\citeyear{Hamana2009} who gave a presentation which
preserves sharing: pointers can not only refer to nodes above them
but also across from them in the cyclic tree. This yields a much more
involved binding structure which would be interesting in its own right.


\bibliography{main}

\end{document}
